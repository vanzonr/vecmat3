%
% vecmat3.tex - LaTeX documentation source for vecmat3.h version 2.3.1: 
%               Fast 3d vector and matrix classes using expression templates
%
% Copyright (c) 2007-2013  Ramses van Zon
%
% Permission is hereby granted, free of charge, to any person obtaining a copy
% of this software and associated documentation files (the "Software"), to deal
% in the Software without restriction, including without limitation the rights
% to use, copy, modify, merge, publish, distribute, sublicense, and/or sell
% copies of the Software, and to permit persons to whom the Software is
% furnished to do so, subject to the following conditions:
%
% The above copyright notice and this permission notice shall be included in
% all copies or substantial portions of the Software.
%
% THE SOFTWARE IS PROVIDED "AS IS", WITHOUT WARRANTY OF ANY KIND, EXPRESS OR
% IMPLIED, INCLUDING BUT NOT LIMITED TO THE WARRANTIES OF MERCHANTABILITY,
% FITNESS FOR A PARTICULAR PURPOSE AND NONINFRINGEMENT. IN NO EVENT SHALL THE
% AUTHORS OR COPYRIGHT HOLDERS BE LIABLE FOR ANY CLAIM, DAMAGES OR OTHER
% LIABILITY, WHETHER IN AN ACTION OF CONTRACT, TORT OR OTHERWISE, ARISING FROM,
% OUT OF OR IN CONNECTION WITH THE SOFTWARE OR THE USE OR OTHER DEALINGS IN
% THE SOFTWARE.
%

\documentclass[12pt,twoside]{article}

\usepackage{a4wide,fancyhdr,mathptm,charter,color}
%\usepackage[colorlinks]{hyperref}

\pagestyle{fancy}
\renewcommand{\sectionmark}[1]{\markright{#1}}
\fancyhf{}
\fancyhead[LE,RO]{\bfseries\thepage}
\fancyhead[LO]{\bfseries \hfill\rightmark\hfill}
\fancyhead[RE]{\hfill\bfseries\rightmark \hfill}
\fancyfoot[L]{\textit{vecmat3} v2.3.1}
\fancyfoot[R]{\hfill{\color{white}.}\hfill Ramses van Zon\hfill May 15, 2013}
\renewcommand{\headrulewidth}{0.5pt}
\renewcommand{\footrulewidth}{0pt}
\addtolength{\headheight}{2.5pt}
\fancypagestyle{plain}{\fancyhead{}\renewcommand{\headrulewidth}{0pt}}


\newcommand{\Vector}{{Vector}}
\newcommand{\Matrix}{{Matrix}}
\newcommand{\cxx}{C\texttt{++}}
\newcommand{\TT}{{\tt<}T{\tt>}}

\begin{document}

\setlength{\parskip}{1mm}

\title{User Documentation for the  \emph{vecmat3}\\\Vector\ and
  \Matrix\ classes\\(version 2.3.1)}

\author{Ramses van Zon\footnote{vanzonr@gmail.com}}

\date{May 15, 2013}

\maketitle
\begin{abstract}
This document describes how to use the \cxx\ \Vector\ and
\Matrix\ classes defined in the header file vecmat3.h. These classes
offer a very convenient notation for three dimensional vector and
matrix algebra by using overloaded operators.  Furthermore, they are
constructed to be numerically efficient through the internal use of
expression templates, without interfering at the user level.  As a
result, one can convert mathematical matrix-vector expressions
straightforwardly into the corresponding \cxx\ expression without
having to worry about incurring an efficiency penalty.
\end{abstract}

\newpage

\renewcommand{\contentsname}{\vspace{-13mm}}\markright{Contents}
\tableofcontents

\newpage

\section{Introduction}
Vectors and matrices are used frequently in scientific computation (as
well as in modeling, games and movie rendering).  Unfortunately, no
built-in support for matrices and vectors exists in \cxx.  In
principle, expressions involving matrices and vectors, such as
\[
\vec a=\vec b+\mathsf M \cdot \vec c
\]
(with $\vec a$, $\vec b$ and $\vec c$ vectors and $\mathsf M$ a
matrix) can be implemented in \cxx\
such that they strongly resemble their mathematical notations, e.g.
\begin{quote}
\tt
Vector a,b;

Matrix M;

Vector c = a + M*b;
\end{quote}
The technique used to accomplish such notational convenience is
operator overloading, whose straightforward implementation comes with
a high computational cost due to the creation of temporary
objects.

A more efficient implementation is possible by using \emph{expression
templates}.  Efficient matrix-vector implementations are somewhat of a
by-product of \cxx\ templates, and this shows in the awkward and
complicated notation needed for general matrix-vector
manipulations. In addition, the \cxx\ standard is somewhat quirky on
what is and is not allowed when using templates.

These notational issues probably explain why there are far fewer
template-based implementations of matrices and vectors available.
This is especially problematic for small vectors and matrices of fixed
size, for instance three-dimensional ones. These allow additional
efficiency gains over general-size vectors and matrices (because loops
over indices can be replaced by explicit sums in the implementation).
Two known implementations are the \texttt{TinyVector} and \texttt{
TinyMatrix} classes of \texttt{Blitz++} and the ones by the same name of
\texttt{tvmet}. The former is not very developed, i.e., many operations
that one would like to have are not present, and indeed, the latter is
aimed at fixing that. Still, \texttt{tvmet} lacks some functionality that
built-in types do have, for instance, \texttt{TinyVector<3,double> v =
a+b;} is not possible.

This is where \emph{vecmat3} comes in. It defines very efficient
three dimensional vector and matrix manipulations.  The aim is to be
able to use these vectors and matrices as if they were built-in types,
with which the same kind of expressions can be formed as can with
built-in types without worrying about template techniques, but also without
substantial losses compared to hard-coded element-by-element
techniques.

\emph{vecmat3} uses expression templates and operator
overloading. The restrictions of vecmat3 at present are that the
elements of the vectors and matrices have to be of a single type,
which is \texttt{double} by default, and that only three-dimensional
quantities are supported (as the name suggests).

\section*{Change history}
\subsection*{Changes in version 2}

The main difference between the first version of vecmat3 and the
second is that the matrices and vectors no longer need to be all of
one type in a single application.  In addition to the standard Vector
and Matrix classes of type \texttt{DOUBLE} (which defaults to double)
as in version 1, in version 2 one also has three-dimensional
structures of any type \texttt{T} at one's disposal. To be more
precise:
\begin{itemize}
\item \texttt{vecmat3::Vector\TT{}} and \texttt{vecmat3::Matrix\TT{}} are
  three-dimensional vector and matrix classes whose elements are of
  type \texttt{T}. 

  For example, one can define a 3x3 matrix of integers as\ \ 
  \texttt{vecmat3::Matrix<int>~m;}
\item This notation reflects two changes in the code: 
\begin{itemize}
\item Almost all of the \texttt{vecmat3} code is now contained in its own
  namespace called \texttt{vecmat3}. 
\item The type is a template argument.
\end{itemize}
\item In version 2 the standard Vector and Matrix classes are simply
  typedef'ed as equivalent to \texttt{vecmat3::Vector<DOUBLE>} and
  \texttt{vecmat3::Vector<DOUBLE>}, whereas in version 1 they were the
  only vectors and matrices available.
\item The typedef's of Vector and Matrix will be omitted if the
  compiler flag \texttt{NOVECMAT3DEF} is defined.
\end{itemize}

\subsection*{Changes between version 2 and version 2.3}

(Versions 2.1 and 2.2 were internal development stages.)

\begin{itemize}
\item Most functionality remained the same as in version 2, except that square bracket support has been added. See Section \ref{bracketsnowtoo}.

\item The header file now also enforces inlining the template functions for
the GNU (tested version 4.4.0) and the Intel compilers (tested
versions 11 and 12), even when no optimization is used.  

\item A major improvement of version 2.3 is that the library is now
compatible with IBM's xlC compiler, which had trouble with some
template constructions in versions 1 and 2. While I'm on the
subject, one should compile with \texttt{-O4} when using the IBM
compilers with vecmat3 in order to get all inlining done properly (in
the latest version, xlC 11, the options \texttt{-O2 -qinline=level=6} suffice).
\end{itemize}

\subsection*{Changes between version 2.3 and version 2.3.1}

Version 2.3.1 comes with an open-source license (MIT). Here is the
text of the license:
\begin{verbatim}
Copyright (c) 2007-2013 Ramses van Zon

Permission is hereby granted, free of charge, to any person obtaining a copy
of this software and associated documentation files (the "Software"), to deal
in the Software without restriction, including without limitation the rights
to use, copy, modify, merge, publish, distribute, sublicense, and/or sell
copies of the Software, and to permit persons to whom the Software is
furnished to do so, subject to the following conditions:

The above copyright notice and this permission notice shall be included in
all copies or substantial portions of the Software.

THE SOFTWARE IS PROVIDED "AS IS", WITHOUT WARRANTY OF ANY KIND, EXPRESS OR
IMPLIED, INCLUDING BUT NOT LIMITED TO THE WARRANTIES OF MERCHANTABILITY,
FITNESS FOR A PARTICULAR PURPOSE AND NONINFRINGEMENT. IN NO EVENT SHALL THE
AUTHORS OR COPYRIGHT HOLDERS BE LIABLE FOR ANY CLAIM, DAMAGES OR OTHER
LIABILITY, WHETHER IN AN ACTION OF CONTRACT, TORT OR OTHERWISE, ARISING FROM,
OUT OF OR IN CONNECTION WITH THE SOFTWARE OR THE USE OR OTHER DEALINGS IN
THE SOFTWARE.
\end{verbatim}

\section{Installation}

To use this header-only library, one merely needs to copy the header
file vecmat3.h to the directory of the source files that include it,
or to a default directory where the compiler will look for header
files (e.g. /usr/local/include).

The \emph{vecmat3} library has been tested with the GNU g++ compiler version
4.4 and up, the Intel C++ compiler version 11 and up and IBM's XL C++
compiler version 10 and up.


\section{Using the \Vector\ and \Matrix\ classes}
\label{use}

\subsection{Classes}

\emph{vecmat3} provides two general template classes within the
namespace \texttt{vecmat3}:

\begin{quote}\tt
  template<typename T> vecmat3::Vector;

  template<typename T> vecmat3::Matrix;
\end{quote}
The template parameter T determines the type of the vector and
matrix elements. Thus, for a vector of integers one uses the type
\texttt{vecmat3::Vector<int>}, while for a matrix of doubles one would
use \texttt{vecmat3::Matrix<double>}.

Since applications often need only one type of vector, a default
vector type a and default matrix type are defined outside the
\texttt{vecmat3} namespace, as follows
\begin{quote}\tt

typedef vecmat3::Vector<DOUBLE> Vector;

typedef vecmat3::Matrix<DOUBLE> Matrix;
\end{quote}
Here, DOUBLE is a predefined macro that should contain the type of the
elements of the default vectors and matrices. Thus, \texttt{Vector v;}
defines a vector with elements of type DOUBLE.

The type DOUBLE can be defined in three ways: 
\begin{enumerate}
\item One can write an
\texttt{\#define DOUBLE <something>} before including the
\texttt{vecmat3.h} header, with \texttt{<something>} replaced by the
desired type (e.g.\ \texttt{float} or \texttt{double});
\item One can give a command line argument to the compiler to define
  DOUBLE to be \texttt{<something>} (e.g.\ \texttt{-DDOUBLE=float} for
  g++);
\item One can do nothing, which makes \texttt{DOUBLE} default to
\texttt{double}.
\end{enumerate}

The definition of DOUBLE and the type definition of Vector and Matrix
in the global namespace does pollute the global namespace, and in many
cases is not wanted. These definitions are omitted if \texttt{NOVECMAT3DEF} is defined.

\subsection{Header file}

To use \emph{vecmat3}, the following general procedure should be
followed: 

\begin{itemize}
  \item If the elements of the vectors and matrices are to have a
  different type than \texttt{double}, first \texttt{\#define} their type as
  \texttt{DOUBLE}, e.g.
  \begin{quote}
    \tt \#define DOUBLE float
  \end{quote}
  The type of the elements of a vector or matrix will be referred to
  as the ``value type'' in this documentation.
  \item Include the header file vecmat3.h:
  \begin{quote}
    \tt \#include "vecmat3.h"
  \end{quote}

  \item The class \Vector\ and the class \Matrix\ are now
  defined and instances these classes can be declared as follows:
    \begin{quote}\tt
      Vector a;

      Matrix R;
    \end{quote}
    \item One can explicitly use any other value type
      than DOUBLE type, e.g.
    \begin{quote}\tt
      vecmat3::Vector<int> a;

      vecmat3::Matrix<int> R;
    \end{quote}
    If vectors and matrices of a specific value type are used a lot in an
    application, it may be useful to typedef them to a shorter
    notation, e.g.
    \begin{quote}\tt
      typedef vecmat3::Vector<int> intVector;

      typedef vecmat3::Matrix<int> intMatrix;

      intVector b;

      intMatrix S;
    \end{quote}
  \item Alternatively, one can have no default global \Vector\ and \Matrix\ class defined, and use only the namespace vecmat3, e.g.
    \begin{quote}\tt
      \#define NOVECMAT3DEF

      \#include "vecmat3.h"      

      vecmat3::Vector<double> a;

      vecmat3::Matrix<double> R;
    \end{quote}

      \item In the above examples, the elements of the vectors and
        matrices are unspecified, and likely contain garbage. In the
        next section, it will be explained how to initialize these
        elements of these
        classes.
    \item Note that currently, operations between matrices and vectors
      of different value types are not supported, even when mathematically
      this would make sense (such as for \texttt{int} and
      \texttt{double}). 
      \item However, it is possible to assign any kind of number to an
        element of any value type, as long as a (standard)
        conversion to that type is known to the c++ compiler.  Thus,
        one may, for instance, assign an integer to an element of a
        vector, or one may multiply a vector by 2 (i.e., one may write
        \texttt{2*v} instead of being
        forced to write \texttt{2.0*v} or \texttt{2.0f*v}).
\end{itemize}


\subsection{Initialization methods}

There are four ways to initialize a \Vector\ or \Matrix, which we will
discuss by example.  In describing the initialization methods, the
default Vector and Matrix types will be used; the arbitrary type
versions \texttt{vecmat3::Vector\TT{}} and
\texttt{vecmat3::Matrix\TT{}} have the same functionality.

\subsubsection{Initialization through constructor parameters}
  Example:
    \begin{quote}\tt
         Vector a(1.1, 3.0, -4.3);

         Matrix R(1, 2, 3,
 
\ \ \ \ \ \ \ \ \ 4, 5, 6,

\ \ \ \ \ \ \ \ \ 7, 8, 9);
    \end{quote}
    defines a \Vector\ \texttt a and \Matrix\ \texttt R with specified
    elements. Note that the first set of three elements given to
    \texttt R comprise the top row of \texttt R, the second set of
    three the middle row and the last set of three the bottom row.

    If fewer than three or nine (for \Vector\ and \Matrix,
    respectively) number are given, the unspecified elements are set
    to zero. Thus, one can define a zero \Vector\ and \Matrix\ simply by
    \begin{quote}\tt
         Vector a(0);

         Matrix R(0);
    \end{quote}

  \subsubsection{Initialization through assignment}
  Example:
    \begin{quote}\tt
         Vector b = a;

         Matrix S = R;
    \end{quote}
    defines a \Vector\ \texttt b with the same elements as \texttt a,
    and a \Matrix\ \texttt S with the elements~as~\texttt R. 

    The right hand sides may also be an expression involving \Vector's
    and Matrices. The allowed expressions are explained in
    sections \ref{operations} and \ref{expressions}.
%\newpage
  \subsubsection{Initialization through a comma separated list}
  Example:
    \begin{quote}  \tt
         Vector a;

	 a = 1.1, 3.0, -4.3;

         Matrix R;

	 R = 1, 2, 3, 

     \ \ \ \ 4, 5, 6, 

     \ \ \ \ 7, 8, 9;
    \end{quote}  
    Note: this is the standard construction for \texttt{Blitz++} and
    \texttt{tvmet}, and is achieved though an overloaded comma
    operator. Not everybody likes overloading the comma operator,
    because it may confuse the user (more than the above methods), and
    it is somewhat less efficient than the method in 3.2.1.

    Furthermore, it is  not possible to use this method in
    the declaration, i.e., one cannot write \texttt{Vector
    a=1.1,3.0,-4.3;} since \cxx\ would consider this a
    declaration of 3.0 and -4.3 as being of type \Vector.
    
    If not enough elements are given in the list, the remaining
    elements are set to zero. Thus, one can write
    \begin{quote}\tt
         b = 1;
    \end{quote}
    to get the vector (1,0,0), and 
    \begin{quote}\tt
         R = 2;
    \end{quote}
    to get the matrix $\left(\begin{array}{ccc}2&0&0\\0&0&0\\0&0&0\end{array}\right)$.

  \subsubsection{Initialization through member functions}
  Example:
    \begin{quote}\tt
         Vector a;

	 a.zero();

         Matrix R;

	 R.zero();
    \end{quote}
    also define a \Vector\ and \Matrix, respectively, with zero
    elements.
    
\noindent
    For a \Matrix, there also is a member function {\tt one()} to turn
    it into an identity matrix:
    \begin{quote}\tt
         Matrix S;

	 S.one();
    \end{quote}
    Furthermore, one can initialize a \Matrix\ also per row or column, e.g.
    \begin{quote}\tt
      R.setRow(0,a);

      R.setRow(1,Vector(0,2,0));

      R.setRow(2,Vector(0,2,-1));

      S.setColumn(0,a);

      S.setColumn(1,Vector(0,2,0));

      S.setColumn(2,Vector(0,2,-1));
    \end{quote}  
    Note that rows and columns are numbered from 0 to 2.


 \subsubsection{Arrays}
     Example:
    \begin{quote}\tt
         Vector a[3];

         Matrix S[3];
    \end{quote}
    Defines arrays of three Vectors and Matrices which are
    non-initialized. One can initialize these arrays as follows
    \begin{quote}\tt
         Vector a[3] = \{ Vector(1,2,3), Vector(3,4,5), Vector(5,6,7) \};

         Matrix S[3] = \{ Matrix(0), Matrix(1,2,3,4,5,6,7,8,9), Matrix(2) \} ;
    \end{quote}

\subsection{Accessing the elements}
\label{bracketsnowtoo}

There are three ways to assess the elements of Vectors and Matrices:

\begin{enumerate}

\item Basic elements of Vectors and Matrices are generally accessible using the
parenthesis notation, i.e., the elements of a  \Vector\ \texttt a are
\texttt{a(0)}, \texttt{a(1)} and \texttt{a(2)}, while those of a
\Matrix\ \texttt R are \texttt{R(0,0)}, \texttt{R(0,1)}, \dots
\texttt{R(2,2)}.

\item Bracket notation can also be used, 
 i.e., the elements of a  \Vector\ \texttt a are
\texttt{a[0]}, \texttt{a[1]} and \texttt{a[2]}, while those of a
\Matrix\ \texttt R are \texttt{R[0][0]}, \texttt{R[0][1]}, \dots
\texttt{R[2][2]}. Note the double brackets for matrix element access.
This way, Vector and Matrix objects act as if they are of type \texttt{T[3]} and \texttt{T[3][3]}, respectively. Accessing the matrix elements in this way may be moderately slower than the parenthesis method (depending on the compiler).

\item Another way to access the elements is though the class members
themselves, i.e., \texttt x, \texttt y and \texttt z for \Vector, and
\texttt{xx}, \texttt{xy}, \texttt{xz}, \texttt{yx}, \texttt{yy},
\texttt{yz}, \texttt{zx}, \texttt{zy} and \texttt{zz} for Matrix. This is
potentially  more efficient, but cannot be used for expressions, i.e.,
\texttt{(A+B).xx} is not possible, unless one writes \texttt{Matrix(A+B).xx}.
\end{enumerate}
Furthermore, the rows and columns of a Matrix can be used as if they
were vectors as follows
\begin{quote}\tt
  Vector v = R.row(1);

  Vector w = R.column(2);
\end{quote}


\subsection{Operators}
\label{operations}


The available algebraic operators for the Vector and Matrix classes
are summarized in table 1, in which '\Vector\TT' stands for
'\texttt{const Vector\TT\&}' or a \Vector-valued expression, and '\Matrix\TT'
stands for '\texttt{const Matrix\TT\&}' or a Matrix\TT-valued expression.

In addition, \texttt{<<} operators are defined for output of Vectors
and Matrices to \texttt{ostream}s, such that
\begin{quote}\tt
  Vector a(1,2,3);

  std::cout << a << endl;
\end{quote}
would print the numbers 1, 2 and 3 with just a space in between.
\begin{quote}\tt
  Matrix<T> M(1,2,3,4,5,6,7,8,9);

  std::cout << M << endl;
\end{quote}
would print a newline, the numbers 1, 2 and 3, another newline, the
numbers 4, 5 and 6, another newline, the numbers 7, 8 and 9 and
finally another newline.

\begin{table}[t]\small
\begin{center}
\begin{tabular}{|rcl|c|c|rcl|}
\hline
 \multicolumn{3}{|c|}{\bf form} & \bf description & \bf example & \multicolumn{3}{c|}{\bf mathematically}\\\hline\hline
%
&\texttt{\textcolor{red}-}
&\texttt{Vector<T>}  
& negative 
& \tt c = -a; 
& $\quad \vec c$&=&$-\vec a$\\
%
\texttt{Vector<T>}
&\textcolor{red}{\texttt +}
&\texttt{Vector<T>} 
& add 
& \tt c = a + b; 
& $\quad \vec c$&=&$\vec a +\vec b$\\
%
\texttt{Vector<T>}
&\textcolor{red}{\texttt -}
&\texttt{Vector<T>} 
& subtract
& \tt c = a - b;
& $\quad \vec c$&=&$\vec a -\vec b$\\
%
\texttt{T}
&\textcolor{red}{\texttt *}
&\texttt{Vector<T>} 
& multiply with scalar 
& \tt c = d * a; 
& $\quad \vec c$&=&$d\,\vec a$\\
%
\texttt{Vector<T>}
&\textcolor{red}{\texttt *} 
&\texttt{T} 
& multiply with scalar 
& \tt c = a * d; 
& $\quad \vec c$&=&$\vec a\, d$\\
%
\texttt{Vector<T>}
&\textcolor{red}{\texttt /}
&\texttt{T} 
& divide by scalar 
& \tt c = a / d; 
& $\quad \vec c$&=&$\vec a / d$\\
%
\texttt{Vector<T>} 
&\textcolor{red}{\texttt \^{}} 
&\texttt{Vector<T>} 
& cross/outer product$^\dagger$
& \tt c = a \^{} b; 
& $\quad \vec c$&=&$\vec a \times \vec b$\\
%
\texttt{Vector<T>}
&\textcolor{red}{\texttt *}
&\texttt{Vector<T>} 
& dot/inner product$^\ddagger$
& \tt d = a * b; 
& $\quad d$&=&$\vec a \cdot \vec b$\\
%
\ \textcolor{red}( \texttt{Vector<T>} 
&\textcolor{red}{\texttt |}
& \texttt{Vector<T>} \textcolor{red})
& dot/inner product$^\ddagger$
& \tt d = (a|b); 
& $\quad d$&=&$\vec a \cdot \vec b$\\
%
\hline
%
&\texttt{\textcolor{red}-}
&\texttt{Matrix<T>}  
& negative 
& \tt T = -S; 
& $\quad \mathsf T$&=&$-\mathsf S$\\
%
\texttt{Matrix<T>}
&\textcolor{red}{\texttt +}
&\texttt{Matrix<T>} 
& add 
& \tt T = S + R; 
& $\quad \mathsf T$&=&$\mathsf S +\mathsf R$\\
%
\texttt{Matrix<T>}
&\textcolor{red}{\texttt -}
&\texttt{Matrix<T>} 
& subtract
& \tt T = S - R;
& $\quad \mathsf T$&=&$\mathsf S -\mathsf R$\\
%
\texttt{T}
&\textcolor{red}{\texttt *}
&\texttt{Matrix<T>} 
& multiply with scalar 
& \tt T = d * S; 
& $\quad \mathsf T$&=&$d\,\mathsf S$\\
%
\texttt{Matrix<T>} 
&\textcolor{red}{\texttt *} 
&\texttt{T} 
& multiply with scalar 
& \tt T = S * d; 
& $\quad \mathsf T$&=&$\mathsf S\, d$\\
%
\texttt{Matrix<T>} 
&\textcolor{red}{\texttt /} 
&\texttt{T} 
& divide by scalar 
& \tt T = S / d; 
& $\quad \mathsf T$&=&$\mathsf S / d$\\
%
\texttt{Matrix<T>}
&\textcolor{red}{\texttt *}
&\texttt{Matrix<T>} 
& matrix-matrix product
& \tt T = S * R; 
& $\quad \mathsf T$&=&$\mathsf S\,\mathsf R$\\
%
\texttt{Matrix<T>}
&\textcolor{red}{\texttt *} 
&\texttt{Vector<T>} 
& matrix-vector product
& \tt c = S * a; 
& $\quad \vec c$&=&$\mathsf S \, \vec a$\\
%
\hline
%
\texttt{Vector<T>}
&\textcolor{red}{\texttt{+=}}
&\texttt{Vector<T>} 
& add 
& \tt c += b; 
& $\quad \vec c$&=&$\vec c +\vec b$\\
%
 \texttt{Vector<T>}
&\texttt{\textcolor{red}{-=}}
&\texttt{Vector<T>} 
& subtract 
& \tt c -= b; 
& $\quad \vec c$&=&$\vec c -\vec b$\\
%
\texttt{Vector<T>}
&\texttt{\textcolor{red}{*=}}
&\texttt{T} 
& multiply by scalar
& \tt c *= d; 
& $\quad \vec c$&=&$d\,\vec c$\\
%
\texttt{Vector<T>}
&\texttt{\textcolor{red}{/=}}
&\texttt{T} 
& divide by scalar
& \tt c /= d; 
& $\quad \vec c$&=&$\vec c/d$\\
%
\texttt{Matrix<T>}
&\texttt{\textcolor{red}{+=}}
&\texttt{ Matrix<T>} 
& add 
& \tt T += R; 
& $\quad \mathsf T$&=&$\mathsf T +\mathsf R$\\
%
 \texttt{Matrix<T>}
&\texttt{\textcolor{red}{-=}}
&\texttt{ Matrix<T>} 
& subtract 
& \tt T -= R; 
& $\quad \mathsf T$&=&$\mathsf T -\mathsf R$\\
%
\texttt{Matrix<T>}&
\texttt{\textcolor{red}{*=}}
&\texttt{T} 
& multiply by scalar
& \tt T *= d; 
& $\quad \mathsf T$&=&$d\,\mathsf T$\\
%
 \texttt{Matrix<T>}
&\texttt{\textcolor{red}{/=}}
&\texttt{T} 
& divide by scalar
& \tt T /= d; 
& $\quad \mathsf T$&=&$\mathsf T/d$\\
\hline
\multicolumn{6}{l}{
$^\dagger$ The \texttt \^{} operator has rather low precedence, so
often one has to write \texttt{(a\^{}b)}.
}\\\multicolumn{6}{l}{
$^\ddagger$ Two operators are provided for the dot product, which do
the exact same thing. 
}
\end{tabular}
\end{center}
\caption{Operators available for matrices and
  vectors with elements of type \texttt{T}. }
\end{table}



\subsection{Member functions}

For any \Vector, or \Vector-valued expression, or for any \Matrix, or
\Matrix-valued expression, the following properties are available as
member functions. Note that below, {\tt T} stands for the typename of the template, while {\tt Vector\TT{}} and {\tt Matrix\TT{}} stand for {\tt vecmat3::Vector\TT{}} and {\tt vecmat3::Matrix\TT{}}, respectively.  The examples all use {\tt T=DOUBLE}.

\subsubsection{T nrm2()}

This returns the sum of the squares of the elements, which is its norm
squared. E.g., with {\tt T=DOUBLE},
\begin{quote} \tt
  Vector a(1,2,3.316625);

  Matrix R(1,2,0,

\ \ \ \ \ \ \ \ \ 2,0,2
	   
\ \ \ \ \ \ \ \	\ 1,1,1);

  DOUBLE d1 = a.nrm2(); // will be equal to 16.0000014

  DOUBLE d2 = R.nrm2(); // will be equal to 16
\end{quote}

\subsubsection{T nrm()}

This returns the norm of a \Vector\ or \Matrix, e.g., with {\tt T=DOUBLE},
\begin{quote} \tt
  DOUBLE d3 = a.nrm(); // will be equal to 4.00000017

  DOUBLE d4 = R.nrm(); // will be equal to 4
\end{quote}

\noindent
The following properties are for Matrices only:

\subsubsection{T tr()}

This returns the trace of a \Matrix, i.e., the sum of its diagonal
elements. E.g., with {\tt T=DOUBLE},
\begin{quote} \tt
  DOUBLE d5 = R.tr(); // will be equal to 2
\end{quote}

\subsubsection{T det()}

This returns the determinant of a \Matrix, e.g., with {\tt T=DOUBLE},
\begin{quote} \tt
  DOUBLE d6 = R.det(); // will be equal to -2
\end{quote}

\subsubsection{Vector\TT{} row(int i)}
This returns the ith row of a \Matrix.

\subsubsection{Vector\TT{} column(int j)}

This returns the jth column of a \Matrix.

\pagebreak[3]
\subsection{Non-member functions}

In the definition of the following non-member functions, the specified
return type are effective ones. E.g. a return type of Matrix may
return a Matrix-Expression when this is more efficient. In any case, it
can be treated as a Matrix in virtually all ways. Likewise, if an
argument is of Matrix type, a Matrix expression is also allowed.

\subsubsection{Matrix\TT{} Transpose(const Matrix\TT{} \& M)}

Returns the transpose of the argument, which is a \Matrix, e.g.
\begin{quote}\tt
  Matrix S = Transpose(R);
\end{quote}

\subsubsection{Matrix\TT{} Inverse(const Matrix\TT{} \& M)}

Returns the inverse of the argument, which is a \Matrix, e.g.
\begin{quote}\tt
  Matrix S = Inverse(R);
\end{quote}

\subsubsection{Matrix\TT{} Rodrigues(const Vector\TT{} \& v)}

Returns the \Matrix-valued rotation matrix for a rotation along the
axis given by the direction of the \Vector\ argument, with the angle
equal to the norm of that \Vector, e.g.
\begin{quote}\tt
  Matrix S = Rodrigues(a);
\end{quote}


\subsubsection{Matrix\TT{} Dyadic(const Vector\TT{} \& a, const Vector\TT{} \& b)}

Returns the \Matrix-valued dyadic product of two arguments which are
\Vector s, e.g.
\begin{quote}\tt
  Matrix S = Dyadic(a,b);
\end{quote}


\subsubsection{Vector\TT{} MTVmult(const Matrix\TT{} \& M, const Vector\TT{} \& v)}

This simply returns Transpose(Matrix)*Vector. 

\subsubsection{T dist(const Vector\TT{} \& a, const Vector\TT{} \& b)}

Returns the length of the difference vector between a and b. This is a
remnant of earlier versions of the \Vector and \Matrix classes, and
barely if at all more efficient than \texttt{(a-b).nrm()}.

\subsubsection{T dist2(const Vector\TT{} \& a, const Vector\TT{} \& b)}

Returns the square length of the difference vector between a and
b. This is a remnant of earlier versions of the Vector and Matrix
classes, and barely if at all more efficient than \texttt{(a-b).nrm2()}.  

\subsubsection{T distwithshift(const Vector\TT{}\&a,const Vector\TT{}\&b,const Vector\TT{}\&s)}

Returns the length of the difference vector between a and b shifted by
s. This is a remnant of earlier versions of the vector and Matrix
classes, and barely if at all more efficient than \texttt{(a+s-b).nrm()}.


\vspace{1cm}\pagebreak[3]
Finally, because the notation \texttt{a*b} and \texttt{a\^{}b} for dot and
cross product may be confusing, the following equivalent alternatives
are defined:
\nopagebreak
\subsubsection{T dotProduct(const Vector\TT{} \& a, const Vector\TT{} \& b)}

Returns the DOUBLE which is the dot, or inner, product of the two
\Vector\ arguments.  It is by definition equal to
\texttt{(\texttt{Vector}|\texttt{Vector})}.
Example, with {\tt T=DOUBLE}:
\begin{quote}\tt
  DOUBLE d = dotProduct(a,b);
\end{quote}

\subsubsection{Vector\TT{} crossProduct(const Vector\TT{} \& a, const Vector\TT{} \& b)}

Returns the \Vector\ which is the cross, or outer, product of the two
\Vector\ arguments.  It is by definition equal to
\texttt{(\texttt{Vector}\^{}\texttt{Vector})}.
Example:
\begin{quote}\tt
  Vector c = crossProduct(a,b);
\end{quote}


\section{Expressions}
\label{expressions}

Using the above elementary operations and functions, complex
expressions can be constructed just as for built-in type such as
\texttt{double}. To be more specific, for all operator expressions in table 1
on page 8, the arguments can be expressions themselves.

For example, one can write,
\begin{quote}\tt
  Vector r[2] = \{ Vector(1,4,5), Vector(2,3,4) \};

  Vector v[2] = \{ Vector(1,0,0), Vector(-1,0,0) \};

  Vector s(7,1,0);

  Matrix A(1,0,0,0,1,0,0,-1,0);

  DOUBLE t =  (r[0]+2*A*(s\^{}r[1]) ) | (v[1]-v[0]);

  // alternatively:

  //DOUBLE t =  dotProduct(r[0]+2*A*crossProduct(s,r[1]), v[1]-v[0]);
\end{quote}

Internally, the expressions are not computed directly via temporaries,
but are computed only upon assignment. As a result, the definitions of
these operators and functions in vecmat3.h is not as simple as
e.g. \texttt{Vector operator+(Vector\&,Vector\&)}.  For that reason,
above we used \texttt{Vector} and \texttt{Matrix} wherever a
\Vector/\Matrix\ or a \Vector/\Matrix\ expression can occur. Never
mind the implementation though, things work as expected.

\newpage
\renewcommand{\refname}{Background references}
\begin{thebibliography}{9}
\bibitem{Lippmann}  
T. Veldhuizen, \emph{Expression Templates} C++ Report, Vol. 7 No. 5 June
1995, pp. 26-31. See also
http://ubiety.uwaterloo.ca/\~{}tveldhui/papers/Expression-Templates/exprtmpl.html.\\
Reprinted in:
S. B. Lippmann (ed.) \emph{C++ Gems} (Cambridge
  University Press, 1998).
\bibitem{VandevoordeJosuttis} D.\ Vandevoorde and N.\ M.\ Josuttis,
  \emph{C++ Templates: The Complete Guide} (Addison-Wesley, Boston,
  2002).
\bibitem{blitz} http://www.oonumerics.org/blitz
\bibitem{tvmet} http://tvmet.sourceforge.net
\end{thebibliography}
\addcontentsline{toc}{section}{\refname}



\end{document}
